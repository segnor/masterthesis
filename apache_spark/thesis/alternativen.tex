\chapter{Alternative Implementierungen der Bibliotheken und Frameworks des BDAS}
\label{chapter:alternative implementierungen}








Wie in den Abbildungen \ref{fig:BDAS1} und \ref{fig:bdas]} ersichtlich ist, existieren auf jeder Ebene des BDAS auch alternative Implementierungen. Einige davon werden im Folgenden kurz vorgestellt. 


\section{Apache Flink}
\label{section:apache flink}





\section{Alternative zu Spark Streaming: Storm}
\label{section:storm}


Storm ist, wie Apache Streaming, ein Framework für Hadoop, bzw. Spark für verteilte Streaming-Anwendungen. Wo Spark ganz klar eine Verbesserung gegenüber Hadoop darstellt und Shark dementsprechend für Hive, ist die Situation bei Storm und Apache Streaming dagegen nicht so klar determinierbar. 

Storm und Spark Streaming unterscheiden sich fundamental in ihren Verarbeitungsmodellen \citeint{va14}. Das erstgenannte Framework verarbeitet eintreffende Events nacheinander, immer genau eines pro Zeitraum. Spark Streaming sammelt im Gegensatz dazu die Events in Mini-Batch-Jobs und verarbeitet sie paketweise zu definierten Zeiträumen nach wenigen Sekunden. Deshalb kann Storm Latenzzeiten von deutlich unter einer Sekunde erreichen, während Spark Streaming eine Latenzzeit von einigen Sekunden aufweist. Diesen Nachteil macht Spark Streaming durch eine sehr gute Fehlertoleranz wett, da die Mini-Batches nach aufgetretenen Fehlern einfach nochmals bearbeitet werden können und die zuvor fehlerhaft ausgeführte Verarbeitung verworfen wird. Treten hingegen bei Storm Fehler auf, wird genau dieser Datensatz nochmals verarbeitet. Dies bedeutet, dass dieser auch mehrfach verarbeitet werden kann. Durch dieses Verhalten lassen sich die beiden Frameworks grob in zwei Einsatzgebiete verteilen:

Storm ist das Framework der Wahl, wenn Wert auf sehr kurze Latenzzeiten gelegt werden muss, hingegen ist es für statusbehaftete Anwendungen durch die Möglichkeit der Mehrfachverarbeitung ungeeignet. Im Umkehrschluss ist Spark Streaming eine gute Wahl, wenn aufgrund der gestreamten Daten eine Statusmaschine aufgebaut werden soll. Dafür müssen hier höhere Latenzzeiten in Kauf genommen werden.     

\section{Alternative zu MLLibs: H2O - Sparkling Water}
\label{section:h20}


BluBlaBlubb


\section{Alternative zu MLLibs: Dato GraphLab Create}
\label{section:h20}


BluBlaBlubb



\subsection{Zusammenfassung}
\label{section:storm}




