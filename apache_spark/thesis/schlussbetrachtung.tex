\chapter{Schlussbetrachtung }
\label{chapter:schlussbetrachtung}


Blablabla 


\section{Zusammenfassung}
\label{section:zusammenfassung}

Blablba

\section{Ausblick}
\label{section:ausblick}

Die bisher verbreiteten Cluster-Computing-Frameworks für Big Data Analytics, allen voran Hadoop, besitzen zwar umfangreiche Funktionen für Parallelverarbeitung und Funktionen für Task-Verteilung und Fehlerrobustheit, doch trotz umfangreicher Zugriffsmechanismen auf die Ressourcen eines Clusters werden die Vorteile der Nutzung von verteiltem Hauptspeicher hier nicht genutzt \citelit{rdd12}. Speziell für Aufgaben, die auf berechnete Zwischenergebnisse zugreifen müsssen, 


RDDs are conceptually similar to views in a database, and persistent RDDs resemble materialized views [28]. However, like DSM systems, databases typically allow fine-grained read-write access to all records, requiring logging of operations and data for fault tolerance and additional overhead to maintain

consistency. These overheads are not required with the coarse-grained transformation model of RDDs.

