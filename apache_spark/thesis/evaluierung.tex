\chapter{Evaluierung der Komponenten und Alternativen }
\label{chapter:evaluierung}

Im vorhergehenden Kapitel wurden die Implementierungen von Prototypen für einzelne Bibliotheken des BDAS vorgestellt. Nachfolgend werden diese Bibliotheken hinsichtlich ihres spezifischen Laufzeitverhaltens untersucht. Zu diesem Zweck wurden für die einzelnen Anwendungsbereiche zunächst Metriken definiert, die Aussagen über die relative Leistungsfähigkeit zulassen. 

Für diese Beurteilung wurden unterschiedliche Umgebungen eingesetzt, wobei hier die relativen Performancewerte gegenüber den absoluten prioritär betrachtet werden.

Zum Einsatz für die Evaluierung kamen für diese Untersuchung unterschiedliche Ansätze für Clusterinfrastrukturen. Es kamen diverse lokale Installationen, sowie ein leistungsfähiges Rechnercluster an der Beuth Hochschule Berlin zum Einsatz.  

Die Tabelle \ref{tab:lokale hardware} gibt eine detaillierte Übersicht über die lokal verwendeten Hardwarekonfigurationen. Alle lokal eingesetzten Maschinen verfügen über SSDs (Solid State Drives) als Festspeicher.

\begin{table}[!ht]
\centering
\begin{tabular}{| p{3cm} | p{2.2cm} |  p{3cm} |  p{1.2cm} | p{3cm} | }
\hline
System & CPU & Anzahl Cores & RAM & OS\\ \hline \hline
MacBook Pro Mid 2014 & Intel i7 & 4 physisch, 8 mit Hyperthreading & 16 GB & Mac OS-X 10.10.2 Yosemite \\ \hline
Apple iMac Mid 2011 & Intel i5 & 4 physisch, 8 mit Hyperthreading & 16 GB & Mac OS-X 10.10.2 Yosemite \\ \hline
Xeon Workstation & Intel Xeon 1230 & 4 physisch, 8 mit Hyperthreading & 32 GB & Windows 8.1,  VirtualBox Instanzen mit CentOS  \\ \hline 
Lenovo ThinkPad 410T & Intel i5 & 4 physisch & 8 GB & Windows 8.1, VirtualBox Instanzen mit CentOS  \\ \hline 

\end{tabular}
\caption{Übersicht der lokal verwendeten Hardware}
	\label{tab:lokale hardware}
\end{table}  

Die Tabelle \ref{tab:cluster} zeigt die Konfiguration (Konfigurationsstand Dezember 2014) des verwendeten Clustersystems an der Beuth Hochschule Berlin. 

\begin{table}[!ht]
\centering
\begin{tabular}{| p{3cm} | p{2.2cm} |  p{3cm} |  p{1.2cm} | p{3cm} | }
\hline
System & CPU & Anzahl Cores & RAM & OS\\ \hline \hline
Master Node & AMD 6320 & 2 * 8  & 128 GB & Debian Linux \\ \hline
Worker Node 1 & AMD 6378 & 4 * 16 & 512 GB &  Debian Linux\\ \hline
Worker Node 2 & AMD 6378 & 4 * 16 & 512 GB &  Debian Linux\\ \hline
Worker Node 3 & AMD 6378 & 4 * 16 & 512 GB &  Debian Linux\\ \hline

\end{tabular}
\caption{Übersicht über das Cluster der Beuth Hochschule Berlin}
	\label{tab:cluster}
\end{table}  

Des weiteren wurden auch funktionale Anforderungen für die Evaluierung der Frameworks und Bibliotheken definiert. Dies hat besondere Relevanz für die möglichst direkten Vergleiche der Frameworks Spark und Flink, sowie MLLib und H2O. 



\section{Definition von Metriken für die Bibliotheken des BDAS}
\label{section:definition der metriken}



Um die verschiedenen Bibliotheken des BDAS möglichst einheitlich und dennoch nutzungsspezifisch testen zu können, wurden im Rahmen dieser Thesis entsprechende Metriken definiert. Diese sind mehrstufig angeordnet, so dass ein vergleichbares Subset über alle Nutzungsfälle angelegt werden konnte. In einer zweiten Schicht sind zusätzliche, anwendungsfallspezifische Metriken angesiedelt. 

Diese Metriken teilen sich zum einen prinzipiell auf in funktionale und nichtfunktionale Anforderungen. Die funktionalen Anforderungen beschreiben den Funktionsumfang und die Umsetzungsstrategie bestimmter Anforderungen. Die nichtfunktionalen Anforderungen betrachten Aspekte wie Performance auf verschiedenen Ausführungsstufen und nach \(1...n\) Iterationen, Replikationsmechanismen, Verhalten im Fehlerfall. Oberhalb von diesen Metriken sind die speziellen anwendungsfallmetriken. Hier werden beispielsweise die Nachrichten-Durchsatzraten von Spark Streaming oder die Güte und Performance der erlernten Modelle von MLLibs und H2O quantifiziert.  



\section{Beschreibung der Messverfahren}
\label{section:messumgebungen}

TBD!


\section{Beschreibung der Messumgebungen}
\label{section:messumgebungen}

TBD!

\subsection{Lokales Single Node Cluster  }
\label{section:lokales single node}

TBD!

\subsection{Lokales Multi Node Cluster}
\label{section:tachyon}

TBD!

\subsection{Remote Cluster an der Beuth Hochschule}
\label{section:remote}

TBD!

\section{Ergebnisse}
\label{section:ergebnisse}

TBD!

\subsection{Messergebnisse Apache Spark}
\label{section:spark eval}

TBD!

\subsection{Messergebnisse Apache Flink}
\label{section:mllib arch}

TBD!

\subsection{Messergebnisse MLLib}
\label{section:mllib arch}

TBD!

\subsection{Messergebnisse H2O}
\label{section:mllib arch}

TBD!



\section{Zusammenfassung}
\label{section:zusammen}



TBD!
