\chapter{Allgemeine Grundlagen }
\label{chapter:allgemeine Grundlagen}


Das nachfolgende Kapitel behandelt die Grundlagen, die für ein Verständnis der Anwendungsbereiche von Apache Spark, dem Berkeley Data Analytics Stack und im Allgemeinen des Themenkomplexes Big Data Analytics nötig sind. Im ersten Unterkapitel werden die grundsätzlichen Eigenschaften eines verteilten Systems beschrieben um die Basis für die in der Arbeit beschriebenen Besonderheiten von Verarbeitungen im Clusterbetrieb zu legen. Hier wird ein exemplarischer Clusteraufbau skizziert, Probleme mit Concurrency und Netzwerkverkehr beschrieben und welche Möglichkeiten es hier gibt.  Im darauf folgenden Unterkapitel werden grundlegende Problemstellungen und Technologien  beschrieben, die im Rahmen von Big Data Analytics im Allgemeinen vorkommen. Unter anderem werden hier Grundlagen und Begriffe aus den Themengebieten Loganalysen, Machine Learning, statistische Analysen, Graph-Suchen, Datenbankabfragen und Streaming-Frameworks in Kurzform erklärt. In einer Zusammenfassung werde diese Grundlagen nochmals auf einen Blick dargestellt. 

\section{Cluster Computing}
\label{section:cluster computing}

Blablba

\section{Statistikanwendungen}
\label{section:cluster computing}

Blablba

\subsection{Log Analysen}
\label{section:cluster computing}

Blablba

\subsection{Messreihen}
\label{section:cluster computing}

Blablba

\section{Machine Learning}
\label{section:cluster computing}

Blablba

\subsection{Vorhersagealgorithmen}
\label{section:cluster computing}

Blablba

\section{Streaming Frameworks}
\label{section:cluster computing}

Blablba

\section{Anwendungen von Graphen}
\label{section:cluster computing}

Blablba