%%
%% Beuth Hochschule für Technik --  Abschlussarbeit
%%
%% Anhang
%%
%%%%%%%%%%%%%%%%%%%%%%%%%%%%%%%%%%%%%%%%%%%%%%%%%%%%%%%%%%%%%%%%%%%%%

\chapter{Zusätze}
\label{chapter:anhang}

\setglossarysection{section}
\printglossary[type=\acronymtype,title=Abkürzungen]


\section{Übersicht der RDD Transformationen}
\label{section:rddtransform}

Die folgende Tabelle zeigt alle Transformationen, die auf Spark RDDs möglich sind (Entnommen aus \citeint{spa15} - Stand Spark 1.2.0):

\begin{tabular}{| p{5cm} | p{8cm} | }
\hline
Transformation & Zweck \\ \hline \hline
map(func) & Return a new distributed dataset formed by passing each element of the source through a function func.  \\ \hline 
filter(func) & Return a new dataset formed by selecting those elements of the source on which func returns true. \\ \hline 
flatMap(func) & Similar to map, but each input item can be mapped to 0 or more output items (so func should return a Seq rather than a single item).\\ \hline 
mapPartitions(func) & Similar to map, but runs separately on each partition (block) of the RDD, so func must be of type Iterator<T> => Iterator<U> when running on an RDD of type T. \\ \hline 
mapPartitionsWithIndex(func) & Similar to mapPartitions, but also provides func with an integer value representing the index of the partition, so func must be of type (Int, Iterator<T>) => Iterator<U> when running on an RDD of type T. \\ \hline 
sample(withReplacement, fraction, seed) & Sample a fraction fraction of the data, with or without replacement, using a given random number generator seed. \\ \hline 
union(otherDataset) & Return a new dataset that contains the union of the elements in the source dataset and the argument. \\ \hline 
intersection(otherDataset) & Return a new RDD that contains the intersection of elements in the source dataset and the argument. \\ \hline 
distinct([numTasks])) & Return a new dataset that contains the distinct elements of the source dataset. \\ \hline 

\end{tabular}

\begin{tabular}{| p{5cm} | p{8cm} | }
\hline
Transformation & Zweck \\ \hline \hline
groupByKey([numTasks]) & When called on a dataset of (K, V) pairs, returns a dataset of (K, Iterable<V>) pairs. 
Note: If you are grouping in order to perform an aggregation (such as a sum or average) over each key, using reduceByKey or combineByKey will yield much better performance. 
Note: By default, the level of parallelism in the output depends on the number of partitions of the parent RDD. You can pass an optional numTasks argument to set a different number of tasks. \\ \hline 
reduceByKey(function, [numTasks]) & When called on a dataset of (K, V) pairs, returns a dataset of (K, V) pairs where the values for each key are aggregated using the given reduce function func, which must be of type (V,V) => V. Like in groupByKey, the number of reduce tasks is configurable through an optional second argument. \\ \hline 
aggregateByKey(zeroValue) (seqOp, combOp, [numTasks]) & When called on a dataset of (K, V) pairs, returns a dataset of (K, U) pairs where the values for each key are aggregated using the given combine functions and a neutral "zero" value. Allows an aggregated value type that is different than the input value type, while avoiding unnecessary allocations. Like in groupByKey, the number of reduce tasks is configurable through an optional second argument. \\ \hline
sortByKey([ascending], [numTasks]) & When called on a dataset of (K, V) pairs where K implements Ordered, returns a dataset of (K, V) pairs sorted by keys in ascending or descending order, as specified in the boolean ascending argument.  \\ \hline
\end{tabular}

\begin{tabular}{| p{5cm} | p{8cm} | }
\hline
Transformation & Zweck \\ \hline \hline
join(otherDataset, [numTasks]) & When called on datasets of type (K, V) and (K, W), returns a dataset of (K, (V, W)) pairs with all pairs of elements for each key. Outer joins are supported through leftOuterJoin, rightOuterJoin, and fullOuterJoin. \\ \hline
cogroup(otherDataset, [numTasks]) & When called on datasets of type (K, V) and (K, W), returns a dataset of (K, Iterable<V>, Iterable<W>) tuples. This operation is also called groupWith. \\ \hline
cartesian(otherDataset) & When called on datasets of types T and U, returns a dataset of (T, U) pairs (all pairs of elements). \\ \hline
pipe(command, [envVars]) & Pipe each partition of the RDD through a shell command, e.g. a Perl or bash script. RDD elements are written to the process's stdin and lines output to its stdout are returned as an RDD of strings. \\ \hline
coalesce(numPartitions) & Decrease the number of partitions in the RDD to numPartitions. Useful for running operations more efficiently after filtering down a large dataset.  \\ \hline
repartition(numPartitions) & Reshuffle the data in the RDD randomly to create either more or fewer partitions and balance it across them. This always shuffles all data over the network.  \\ \hline
repartitionAndSortWithin Partitions(partitioner) & Repartition the RDD according to the given partitioner and, within each resulting partition, sort records by their keys. This is more efficient than calling repartition and then sorting within each partition because it can push the sorting down into the shuffle machinery.  \\ \hline


\end{tabular}

Die folgende Tabelle zeigt alle Actions, die auf Spark RDDs möglich sind (Entnommen aus \citeint{spa15} - Stand Spark 1.2.0):


\begin{tabular}{| p{5cm} | p{8cm} | }
\hline
Actions & Zweck \\ \hline \hline
reduce(func) & Aggregate the elements of the dataset using a function func (which takes two arguments and returns one). The function should be commutative and associative so that it can be computed correctly in parallel.  \\ \hline 
collect() & Return all the elements of the dataset as an array at the driver program. This is usually useful after a filter or other operation that returns a sufficiently small subset of the data. \\ \hline 
count() & Return the number of elements in the dataset. \\ \hline 
first() & Return the first element of the dataset (similar to take(1)). \\ \hline 
take(n) & Return an array with the first n elements of the dataset. Note that this is currently not executed in parallel. Instead, the driver program computes all the elements.  \\ \hline 
takeSample(withReplacement, num, [seed]) & Return an array with a random sample of num elements of the dataset, with or without replacement, optionally pre-specifying a random number generator seed. \\ \hline 
takeOrdered(n, [ordering]) & Return the first n elements of the RDD using either their natural order or a custom comparator. \\ \hline 
saveAsTextFile(path) & Write the elements of the dataset as a text file (or set of text files) in a given directory in the local filesystem, HDFS or any other Hadoop-supported file system. Spark will call toString on each element to convert it to a line of text in the file. \\ \hline 

\end{tabular}

\begin{tabular}{| p{5cm} | p{8cm} | }
\hline
Actions & Zweck \\ \hline \hline
saveAsSequenceFile(path) & Write the elements of the dataset as a Hadoop SequenceFile in a given path in the local filesystem, HDFS or any other 
Hadoop-supported file system. This is available on RDDs of key-value pairs that either implement Hadoop's Writable interface. In Scala, it is also available on types that are implicitly convertible to Writable (Spark includes conversions for basic types like Int, Double, String, etc). \\ \hline
saveAsObjectFile(path) & Write the elements of the dataset in a simple format using Java serialization, which can then be loaded using SparkContext.objectFile().\\ \hline 
countByKey() & Only available on RDDs of type (K, V). Returns a hashmap of (K, Int) pairs with the count of each key.  \\ \hline
foreach(func) & Run a function func on each element of the dataset. This is usually done for side effects such as updating an accumulator variable (see below) or interacting with external storage systems.\\ \hline 
\end{tabular}
